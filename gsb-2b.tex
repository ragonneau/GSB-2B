%% gsb-2b.tex
%% Copyright 2022 Tom M. Ragonneau
%
% This work may be distributed and/or modified under the
% conditions of the LaTeX Project Public License, either version 1.3
% of this license or (at your option) any later version.
% The latest version of this license is in
%   http://www.latex-project.org/lppl.txt
% and version 1.3 or later is part of all distributions of LaTeX
% version 2005/12/01 or later.
%
% This work has the LPPL maintenance status `maintained'.
%
% The Current Maintainer of this work is Tom M. Ragonneau.
\documentclass[10pt]{article}
\usepackage[%
    a4paper,%
]{geometry}
\usepackage[dvipsnames]{xcolor}
\usepackage[american]{babel}
\usepackage{csquotes}
\usepackage{array}
\usepackage{tabularx}

% Latin fonts
\usepackage[T1]{fontenc}
\usepackage{fontspec}
\setmainfont{TeX Gyre Pagella}

% Bibliography information
\usepackage[%
    style=numeric-comp,%
    sorting=nyt,%
    sortcites,%
]{biblatex}
\addbibresource{ragonneau-bib/strings.bib}
\addbibresource{ragonneau-bib/optim.bib}

% Mathematics
\usepackage{amsmath}
\usepackage{dsfont}
\newcommand{\R}{\ensuremath{\mathds{R}}}

% Cross-referencing and colorization
\usepackage{hyperref}
\usepackage{url}
\hypersetup{%
    colorlinks=true,%
    linkcolor=OliveGreen,%
    anchorcolor=black,%
    citecolor=MidnightBlue,%
    filecolor=black,%
    menucolor=black,%
    runcolor=black,%
    urlcolor=black,%
    linktoc=page,%
}

% Automatic references
\usepackage[noabbrev]{cleveref}
\crefname{equation}{}{}
\Crefname{equation}{}{}

% Spacing between lines
\usepackage{setspace}
\onehalfspacing

% List of hyphenation exceptions for US English
% Source: https://ctan.org/tex-archive/info/digests/tugboat/hyphenex
\input{ushyphex}

% Redefine \maketitle command and set the end of the document
\AtBeginDocument{%
    \renewcommand{\maketitle}{%
        \begin{center}%
        {\LARGE\bfseries\thetitle\par}%
            \vspace{2pc}%
        \end{center}%
    }%
}
\AtEndDocument{%
    \vfill%
    \begingroup%
    \setlength{\extrarowheight}{1pc}%
    \noindent\begin{tabularx}{\textwidth}{@{}lX@{}}%
                 Signature      & \\ \cline{2-2}%
                 Student name   & \textit{\theauthor} \\ \cline{2-2}%
                 Date           & \textit{\thedate} \\ \cline{2-2}%
    \end{tabularx}%
    \endgroup%
}

% Document metadata
\usepackage{titling}
\title{Brief description of the thesis}
\author{Tom M. Ragonneau}
\date{\today}
\hypersetup{%
    pdftitle=\thetitle,%
    pdfauthor=\theauthor,%
    pdfsubject={},%
    pdfkeywords={},%
}

\begin{document}

    \maketitle

    \noindent\textbf{Thesis title} ---
    Model-Based Derivative-Free Optimization Methods and Software.

    \vspace{1pc}

    \noindent\textbf{Brief description} ---
    The vast majority of the existing optimization algorithms rely on classical or generalized derivative information.
    However, such information is unavailable or prohibitively expensive to assess in numerous applications, especially when the problems are derived from experiments or simulations, such as in machine learning~\cite{Ghanbari_Scheinberg_2017}.
    Derivative-free optimization (DFO) is designed to tackle such problems~\cite{Audet_Hare_2017,Conn_Scheinberg_Vicente_2009b}, and this thesis investigates various numerical methods for solving DFO problems.
    It is split into two parts.
    We first present PDFO, a software we developed for interfacing Powell's DFO solvers.
    The second part of this thesis introduces COBYQA, a new method for solving nonlinearly-constrained DFO problems.

    Late Professor M.\ J.\ D.\ Powell designed five DFO methods for unconstrained and constrained problems, namely UOBYQA~\cite{Powell_2002}, NEWUOA~\cite{Powell_2006}, BOBYQA~\cite{Powell_2009}, LINCOA, and COBYLA~\cite{Powell_1994}.
    He implemented them in Fortran 77, which is not convenient for most users.
    Hence, we developed PDFO\footnote{Available at \url{https://www.pdfo.net/}.}, a MATLAB and Python software for using Powell's DFO solvers.
    PDFO provides slightly modified versions of Powell's solvers, as we patched the original Fortran code to avoid infinite cycling and memory errors.
    We detected such errors for ill-conditioned problems during the initial development and testing of PDFO\@.

    COBYLA, designed to tackle nonlinearly-constrained problems, uses linear models of the objective and constraint functions.
    We hence developed in the second part of this thesis COBYQA\footnote{Available at \url{https://cobyqa.readthedocs.io/}.}, a method that uses quadratic models, to attempt to overtake the performances of COBYLA\@.
    It is a derivative-free method designed to solve the nonlinearly-constrained problem
    \begin{subequations}
        \label{eq:nlcp}
        \begin{align}
            \min        & \quad f(x) \label{eq:nlcp-obj}\\
            \text{s.t.} & \quad c_i(x) \le 0, ~ i \in \mathcal{I}, \label{eq:nlcp-cub}\\
                        & \quad c_i(x) = 0, ~ i \in \mathcal{E}, \label{eq:nlcp-ceq}\\
                        & \quad l \le x \le u, \label{eq:nlcp-bds}\\
                        & \quad x \in \R^n, \nonumber
        \end{align}
    \end{subequations}
    where the objective and constraint functions~$f$ and~$c_i$, with~$i \in \mathcal{I} \cup \mathcal{E}$, are real-valued functions on~$\R^n$, and the bounds~$l \in (\R \cup \{-\infty\})^n$ and~$u \in (\R \cup \{+\infty\})^n$ satisfy~$l < u$.
    In opposition to the general constraints~\cref{eq:nlcp-cub,eq:nlcp-ceq}, we assume that the bound constraints~\cref{eq:nlcp-bds} cannot be violated, as they often represent inalienable physical or theoretical constraints.
    Hence, COBYQA always respects the bound constraints~\cref{eq:nlcp-bds}.

    The mathematical framework behind COBYQA is a derivative-free variation of the sequential quadratic programming method.
    The convergence properties of COBYQA are globalized by embedding it in a trust-region framework.
    The resulting trust-region subproblem is solved iteratively, using Byrd-Omojokun-like composite steps to address its possible infeasibility.
    The quadratic models maintained by COBYQA are based on the derivative-free symmetric Broyden update~\cite[\S 3.6]{Fletcher_1987}.
    The normal and tangential subproblems are solved numerically using constrained variations of the truncated conjugate gradient method, designed to preserve the bound constraints~\cref{eq:nlcp-bds}.
    A Python implementation of COBYQA has been carefully developed and is publicly available.

    \printbibliography

\end{document}
